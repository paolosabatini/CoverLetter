\documentclass[12pt,a4paper]{article}

\usepackage{charter}
\usepackage[T1]{fontenc}
\usepackage[utf8]{inputenc}
\usepackage[english]{babel}
%\usepackage[binding=5mm]{layaureo}         % margini ottimizzati per l'A4; rilegatura di 5 mm
\usepackage[font=footnotesize,labelfont=bf]{caption}
\usepackage{amsmath}
\usepackage{amssymb}
\usepackage{graphicx,floatflt}
\usepackage{xcolor}
\usepackage{color}
\usepackage{tikz}
\usepackage[margin=1.5cm]{geometry}
\usepackage{xcolor}
\usepackage[colorlinks=true, linkcolor=blue,urlcolor=blue]{hyperref}

\makeatletter
\renewcommand{\maketitle}{\bgroup\setlength{\parindent}{0pt}
\begin{flushleft}
  \textbf{\@title}

  \@author
\end{flushleft}\egroup
}
\makeatother


\pagenumbering{gobble}
\title{\large \textbf{Search for the four-tops production process in the single-lepton and dileptonic opposite-sign channels with the ATLAS experiment}}
\author{\vspace{5mm}\textbf{Authors}: Clara \textsc{Nellist}, Thomas \textsc{Peiffer}, Arnulf \textsc{Quadt}, \underline{Paolo \textsc{Sabatini}}, Elizaveta \textsc{Shabalina}\\[2mm]
{\scriptsize
\hspace{1.8cm} II. Physikalisches Institut\\[0.5mm]
\hspace{1.8cm} Georg-August-Universität Göttingen\\[0.5mm]
\hspace{1.8cm} Friedrich-Hund-Platz 1\\[0.5mm]
\hspace{1.8cm} 37077 Göttingen, Germany\\[0.5mm]
%\hspace{1.6cm} \href{mailto:paolo.sabatini@phys.uni-goettingen.de}{\color{blue!50!black}paolo.sabatini@phys.uni-goettingen.de}
}
}


\date{}

\begin{document}
\maketitle
{ \setlength{\parindent}{0cm}
\begin{small}

The exceptional dataset of $150$ fb$^{-1}$ that is planned to be available by the end of Run II paves the way to precision measurements, in particular in the top sector. One of those is certainly the production of $t\bar{t}t\bar{t}$, an ultra-rare process -- $\sigma_{t\bar{t}t\bar{t}}\,\approx\,10$ fb -- that is not measured yet and very sensitive to many Beyond-Standard-Model (BSM) scenarios. A precise measurement of this process is not only an important test for the Standard Model in a very hadronic environment, but also could put constraints on BSM parameters. \\

The large number of top quarks in the final translates into many possible detection channels: full hadronic, single-leptonic, dileptonic (same- and opposite-sign final state) and multileptonic channels. The single-leptonic and the dileptonic opposite-sign channels address the same challenge: extract a tiny signal in regions completely dominated by $t\bar{t}$ process with high QCD radiation.\\

For this purpose, many techniques are tested: from data-driven estimation for $t\bar{t}$ background to Multivariate analysis for event reconstruction and/or signal discrimination. In this talk an overview over the analysis with a focus on these techniques is given.



\end{small}


}
\end{document}
